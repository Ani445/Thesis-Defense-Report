\section{Strengths of the Approach}
\label{sec:strengths_of_approach}

The proposed UniCL-AffSeg framework offers several notable strengths that position it as a promising advancement in weakly supervised semantic segmentation (WSSS). By integrating the UniCL framework with a Swin Transformer backbone and affinity-based refinement strategies, the approach leverages multi-modal learning, hierarchical feature extraction, and semantic consistency to overcome common limitations in WSSS pipelines. The following subsections outline these key strengths.

\subsection{Multi-Modal Learning and Unified Objectives}
UniCL's unified contrastive and classification objectives enable robust alignment between image and text modalities, enhancing feature representations without requiring dense annotations. This multi-modal synergy facilitates better image-text correspondence, leading to more discriminative class activation maps (CAMs) compared to traditional single-modal methods. As a result, the framework can generalize across diverse datasets and handle complex scenes with multiple objects, reducing reliance on external cues like saliency maps.

\subsection{Hierarchical Feature Extraction with Swin Transformer}
The Swin Transformer's hierarchical architecture provides multi-scale feature maps that capture both local details and global context, making it particularly suited for dense prediction tasks like segmentation. Unlike global attention mechanisms in ViT-based models, Swin's shifted-window design ensures computational efficiency while preserving fine-grained local information, such as object boundaries and textures. This locality bias improves boundary precision in CAMs, addressing a common weakness in WSSS where activations often miss intricate object parts.

\subsection{Affinity-Based Refinement for Semantic Consistency}
The affinity-based CAM refinement technique, which combines encoder-derived affinity maps with decoder affinities, promotes semantic coherence by propagating information across pixel regions. This approach mitigates noise and fragmentation in initial CAMs, leading to more reliable pseudo-labels. By selecting high-quality affinity maps through deviation scoring, the method ensures stability and reduces overfitting, resulting in improved segmentation quality without additional supervision.

\subsection{Efficiency and Scalability}
The overall framework is computationally efficient, benefiting from Swin's linear complexity and UniCL's pre-trained multi-modal representations. This allows for scalable training on large datasets like PASCAL VOC 2012, making the approach feasible for real-world applications with limited resources. Furthermore, the integration of Pixel-Adaptive Refinement (PAR) enhances boundary accuracy by incorporating color and spatial similarities, providing a lightweight yet effective post-processing step.

In summary, these strengths collectively enable UniCL-AffSeg to generate higher-quality CAMs and pseudo-labels, narrowing the gap between weakly supervised and fully supervised segmentation while maintaining practicality. Future iterations could build on these advantages to further optimize performance in challenging scenarios.