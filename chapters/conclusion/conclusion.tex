\chapter{Conclusion}
\label{chap:conclusion}

Semantic segmentation remains a core challenge in computer vision, requiring pixel-level understanding of images to support applications ranging from autonomous driving to medical imaging. While fully supervised approaches have achieved impressive results, they come with the heavy burden of requiring vast amounts of finely annotated data. Weakly Supervised Semantic Segmentation (WSSS) presents a promising alternative by dramatically reducing this annotation effort, relying instead on weaker forms of supervision such as image-level labels. However, the success of WSSS critically depends on the quality of Class Activation Maps (CAMs), which are used to generate pseudo-labels for training segmentation models.

In this thesis, we addressed the core limitations of existing WSSS pipelines: the generation of incomplete, sparse, and coarse CAMs. Recognizing that a backbone's ability to capture both local and global contexts significantly influences CAM quality, we systematically explored stronger backbone architectures to generate better CAMs. Our work did not restrict itself to any single model; instead, we focused on understanding how models such as UniCL and Swin Transformer, with their multi-modal and hierarchical characteristics, could enhance CAM generation. We demonstrated that backbones capable of capturing fine-grained local details along with broader global semantics produce richer and more complete activation maps.

Beyond backbone selection, we studied and experimented on an enhanced CAM refinement strategy. We have combined the strengths of both affinity-based and pixel-wise refinement methods to create a more robust pipeline. But after training we have our results to be very poor. The reason we have identified is that the process aggregating the intermediate features from the UniCL backbone, which were non-uniform in shapes, was not properly handled. Also, for the same reason, the pixel-wise refinement did not work properly. But we have shown that UniCL is a very strong backbone and we have shown that it can localize the objects very well and the CAMs are significantly good. We need to further investigate the affinity-based refinement strategy and how to properly aggregate the features from the backbone. We also need to investigate how to properly use the pixel-wise refinement strategy. 

The importance of this research lies not only in the immediate improvements observed in segmentation results but also in the broader implications for future WSSS systems. By showing that careful selection of backbones and effective affinity-based refinement strategies can close the gap between weakly and fully supervised methods, we offer a viable path toward scalable, annotation-efficient semantic segmentation. Our findings encourage further research into leveraging the rich affinity information inherently captured by advanced architectures and suggest that even stronger results are possible with newer, self-supervised, or multi-modal vision models.

In conclusion, this thesis highlights that improving the initial CAM quality and designing robust refinement pipelines are pivotal to advancing WSSS performance. 