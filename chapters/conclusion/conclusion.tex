\chapter{Conclusion}
\label{sec:conclusion}

This study presented UniCL-AffSeg, a weakly supervised semantic segmentation framework that integrates multi-modal contrastive learning with hierarchical transformer architectures to generate more accurate and spatially coherent Class Activation Maps (CAMs). Building upon UniCL's unified vision-language representations and Swin Transformer's hierarchical design, the proposed method effectively improves pseudo-label quality through affinity-based and pixel-adaptive refinement strategies.

Comprehensive experiments on the PASCAL VOC 2012 dataset demonstrated that UniCL-AffSeg achieves strong segmentation performance under purely weak supervision, with a mean IoU of 50.3 \% on the validation set and 50.8 \% on the test set. The framework excels for large, visually distinctive classes such as bus, sheep, and horse, while highlighting the persistent challenge of segmenting fine-structured or underrepresented categories like person and chair. These findings confirm the value of integrating multi-modal pretraining and affinity-driven refinement for scalable, annotation-efficient segmentation.

While the discussion chapter elaborated on the framework's strengths and limitations, the results collectively emphasize that improving CAM generation and bias mitigation in pretraining remain the most critical next steps toward closing the gap between weakly and fully supervised methods.

The following section outlines several promising directions for advancing this research—such as exploring diverse backbones, hybrid local-global reasoning, and enhanced prompt learning—which together provide a roadmap for future development in weakly supervised and multi-modal semantic segmentation.

\section{Future Work}
\label{sec:future}


So far, we have explored various models, architectures, and methods for refining CAMs, pseudo-labels, and segmentation maps. However, there remain many promising directions for future research and improvement. Below is a comprehensive list of potential future works, combining our previous ideas with additional suggestions from recent advances in WSSS, CLIP, and UniCL:

\begin{itemize}
    \item \textbf{Experiment with Diverse Backbones and Architectures:} Explore a wider range of backbone networks (e.g., ViT, ResNet variants, Mix Transformer, etc.) and decoder architectures to improve CAM and segmentation quality.
    \item \textbf{Advanced Feature Aggregation:} Develop improved methods for aggregating feature maps and attention maps from intermediate layers, such as transformer-based fusion modules or multi-scale feature alignment.
    \item \textbf{Enhanced CAM and Pseudo-Label Refinement:} Investigate new strategies for refining CAMs and pseudo-labels, including graph-based, CRF-based, or affinity-based refinement, as well as uncertainty-aware or region-based approaches.
    \item \textbf{Alternative Segmentation Models:} Evaluate different segmentation heads and decoders (beyond SegFormer) to assess their impact on weakly supervised segmentation performance.
    \item \textbf{Novel Objective Functions:} Explore new loss functions, such as region-based, contrastive, or uncertainty-aware objectives, to better guide the learning process.
    \item \textbf{Prompt Engineering for Multi-Modal Models:} Design and experiment with improved or learnable text prompts for CLIP/UniCL to enhance image-text alignment, especially for rare or ambiguous classes.
    \item \textbf{Self-Supervised and Semi-Supervised Pretraining:} Utilize large-scale self-supervised or semi-supervised pretraining to improve backbone representations before WSSS training.
    \item \textbf{Hybrid Local-Global Reasoning:} Combine local affinity (e.g., Swin Transformer) with global attention mechanisms (e.g., ViT) to improve semantic propagation across distant regions.
    \item \textbf{Cross-Dataset Generalization:} Evaluate and adapt models for better transferability across datasets with different distributions or label sets.
    \item \textbf{Active Learning Integration:} Incorporate active learning strategies to select the most informative samples for annotation, further reducing annotation costs.
    \item \textbf{Uncertainty Estimation and Correction:} Integrate uncertainty modeling to identify and correct unreliable pseudo-labels during training.
    \item \textbf{Explainability and Trustworthiness:} Develop methods to make CAMs and segmentation predictions more interpretable and trustworthy for real-world applications.
    \item \textbf{Efficient Inference and Deployment:} Optimize models for faster inference and lower memory usage, enabling deployment on edge devices or in real-time scenarios.
    \item \textbf{New Refinement Strategies:} Try alternative pseudo-label and segmentation map refinement methods, including advanced CRF variants or graph-based approaches.
\end{itemize}

These directions provide a comprehensive roadmap for future exploration and improvement in weakly supervised semantic segmentation and multi-modal learning.