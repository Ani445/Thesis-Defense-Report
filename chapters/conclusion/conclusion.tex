\chapter{Conclusion}
\label{chap:conclusion}

This study presented UniCL-AffSeg, a weakly supervised semantic segmentation framework that integrates multi-modal contrastive learning with hierarchical transformer architectures to generate more accurate and spatially coherent Class Activation Maps (CAMs). Building upon UniCL's unified vision-language representations and Swin Transformer's hierarchical design, the proposed method effectively improves pseudo-label quality through affinity-based and pixel-adaptive refinement strategies.

Comprehensive experiments on the PASCAL VOC 2012 dataset demonstrated that UniCL-AffSeg achieves strong segmentation performance under purely weak supervision, with a mean IoU of 50.3 \% on the validation set and 50.8 \% on the test set. The framework excels for large, visually distinctive classes such as bus, sheep, and horse, while highlighting the persistent challenge of segmenting fine-structured or underrepresented categories like person and chair. These findings confirm the value of integrating multi-modal pretraining and affinity-driven refinement for scalable, annotation-efficient segmentation.

While the discussion chapter elaborated on the framework's strengths and limitations, the results collectively emphasize that improving CAM generation and bias mitigation in pretraining remain the most critical next steps toward closing the gap between weakly and fully supervised methods.

The following section outlines several promising directions for advancing this research—such as exploring diverse backbones, hybrid local-global reasoning, and enhanced prompt learning—which together provide a roadmap for future development in weakly supervised and multi-modal semantic segmentation.

\section*{Future Work}
\label{sec:future}

So far, we have only studied different models, architectures and methods for refining CAMs,pseudo-labels and segmentation maps. But we have not yet performed enough experiments to find any better approach to generate better CAMs. And the few that we have done showed very poor results. So, we have to do more experiments to find better backbones/approaches to generate better CAMs. We can also try to find better methods to refine the CAMs and pseudo-labels and segmentation maps.

Here is list of things we can do in the future:

\begin{itemize}
    \item Use different backbones and architectures to generate better CAMs.
    \item Look for better methods to aggregate the feature maps and attention maps from the intermediate layers of the backbones.
    \item Use different methods to refine the CAMs and pseudo-labels.
    \item Use different models (so far only SegFormer \cite{fsss_segformer} has been used) to generate better segmentation maps.
    \item Use different methods to refine the segmentation maps.
    \item Look for any other objective functions that can be used to train the models.
    \item 
\end{itemize}

These directions provide a roadmap for future exploration and improvement in our work.