
\section{Per-Class Performance Analysis}
\label{sec:per_class_performance_analysis}

To gain a deeper understanding of our model’s behavior, we analyzed the per-class Intersection over Union (IoU) scores on the PASCAL VOC 2012 validation set. This analysis highlights how the model performs across different object categories, providing insights beyond the overall metrics.

\subsection{High-Performing Classes}

Classes with a large spatial extent and higher representation in the dataset achieve relatively high IoU scores. For instance, \textbf{background} (77.8\%), \textbf{bus} (67.5\%), \textbf{cat} (65.9\%), and \textbf{dog} (65.1\%) are segmented effectively. These results indicate that the model can reliably capture dominant objects with distinctive visual features.

\subsection{Low-Performing Classes}

In contrast, smaller or less frequent classes such as \textbf{person} (16.9\%), \textbf{chair} (25.3\%), and \textbf{potted plant} (28.3\%) exhibit lower IoU scores. This suggests that the model struggles with rare or small objects, likely due to their limited representation in the training set and small spatial footprint in the images.

\subsection{Observations}

Overall, the model performs better on large and visually distinctive classes such as animals and vehicles, while it underperforms on humans, furniture, and small objects. These per-class performance results provide valuable insights that can guide future improvements, such as incorporating multi-scale features or class-specific data augmentation to better handle challenging classes.


\begin{table}[ht]
\centering
\caption{Per-Class IoU Performance on PASCAL VOC (Latest Results)}
\begin{tabular}{|c|l|c|}
\hline
\textbf{Index} & \textbf{Class}      & \textbf{IoU} \\ \hline
0  & background   & 0.7784  \\
1  & aeroplane    & 0.5685  \\
2  & bicycle      & 0.3191  \\
3  & bird         & 0.6044  \\
4  & boat         & 0.4360  \\
5  & bottle       & 0.3677  \\
6  & bus          & 0.6753  \\
7  & car          & 0.4881  \\
8  & cat          & 0.6591  \\
9  & chair        & 0.2525  \\
10 & cow          & 0.5887  \\
11 & diningtable  & 0.4041  \\
12 & dog          & 0.6511  \\
13 & horse        & 0.6132  \\
14 & motorbike    & 0.5846  \\
15 & person       & 0.1697  \\
16 & pottedplant  & 0.2828  \\
17 & sheep        & 0.6305  \\
18 & sofa         & 0.4661  \\
19 & train        & 0.5642  \\
20 & tvmonitor    & 0.4032  \\ \hline
\end{tabular}
\end{table}

\begin{table}[ht]
\centering
\caption{Overall Performance Metrics}
\begin{tabular}{|l|c|}
\hline
\textbf{Metric} & \textbf{Value} \\ \hline
Pixel Accuracy (pAcc) & 0.7981 \\
Mean Accuracy (mAcc)  & 0.6997 \\
Mean IoU (mIoU)       & 0.5003 \\ \hline
\end{tabular}
\end{table}

% The model achieved a mean IoU of 0.5003 on the PASCAL VOC dataset, with a pixel accuracy of 0.7981 and a mean accuracy of 0.6997. Performance varies significantly across classes: large and well-represented classes such as \textbf{background} (0.7784), \textbf{bus} (0.6753), \textbf{cat} (0.6591), and \textbf{dog} (0.6511) are segmented effectively, while smaller or less frequent classes like \textbf{person} (0.1697), \textbf{chair} (0.2525), and \textbf{potted plant} (0.2828) remain challenging. These results suggest that while the model captures the dominant classes well, further improvements are needed to better handle rare and small objects.