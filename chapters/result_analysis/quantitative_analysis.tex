\section{Quantitative Analysis}
\label{sec:quantitative_analysis}

We evaluated the performance of our proposed method on the PASCAL VOC 2012 \textbf{validation} and \textbf{test} sets using standard metrics: \textbf{mean Intersection over Union (mIoU)}, \textbf{pixel accuracy (pAcc)}, and \textbf{mean accuracy (mAcc)}. The results are summarized in Table~\ref{tab:summarized_results}. 

On the \textbf{validation set}, our method achieves a \textbf{mean IoU of 50.3\%}, a pixel accuracy of 80.1\%, and a mean accuracy of 69.99\%.  
On the \textbf{test set}, the model achieves a slightly higher \textbf{mean IoU of 50.8\%}, with a pixel accuracy of 79.6\% and mean accuracy of 71.36\%.  

These results indicate that the model generalizes well across both validation and test sets, consistently segmenting dominant classes effectively. In particular, the strong performance on large and distinctive categories such as \textbf{background}, \textbf{bus}, \textbf{sheep}, and \textbf{cow} highlights the model's ability to capture prominent visual structures. However, performance remains weaker on small or less frequent categories such as \textbf{person}, \textbf{chair}, and \textbf{bicycle}, suggesting that future improvements should focus on enhancing recognition of rare and small objects.


\subsection{Per-Class Performance Analysis}
\label{subsec:per_class_performance_analysis}

To gain a deeper understanding of our model’s behavior, we analyzed the per-class Intersection over Union (IoU) scores on the PASCAL VOC 2012 validation set. This analysis highlights how the model performs across different object categories, providing insights beyond the overall metrics.

\subsubsection{High-Performing Classes}

Classes with a large spatial extent and higher representation in the dataset achieve relatively high IoU scores. For instance, \textbf{background} (77.8\%), \textbf{bus} (67.5\%), \textbf{cat} (65.9\%), and \textbf{dog} (65.1\%) are segmented effectively. These results indicate that the model can reliably capture dominant objects with distinctive visual features.

\subsubsection{Low-Performing Classes}

In contrast, smaller or less frequent classes such as \textbf{person} (16.9\%), \textbf{chair} (25.3\%), and \textbf{potted plant} (28.3\%) exhibit lower IoU scores. This suggests that the model struggles with rare or small objects, likely due to their limited representation in the training set and small spatial footprint in the images.

\subsubsection{Observations}

Overall, the model performs better on large and visually distinctive classes such as animals and vehicles, while it underperforms on humans, furniture, and small objects. These per-class performance results provide valuable insights that can guide future improvements, such as incorporating multi-scale features or class-specific data augmentation to better handle challenging classes.


\begin{table}[ht]
\centering
\caption{Per-Class IoU Performance on PASCAL VOC (Latest Results)}
\begin{tabular}{|c|l|c|}
\hline
\textbf{Index} & \textbf{Class}      & \textbf{IoU} \\ \hline
0  & background   & 0.7784  \\
1  & aeroplane    & 0.5685  \\
2  & bicycle      & 0.3191  \\
3  & bird         & 0.6044  \\
4  & boat         & 0.4360  \\
5  & bottle       & 0.3677  \\
6  & bus          & 0.6753  \\
7  & car          & 0.4881  \\
8  & cat          & 0.6591  \\
9  & chair        & 0.2525  \\
10 & cow          & 0.5887  \\
11 & diningtable  & 0.4041  \\
12 & dog          & 0.6511  \\
13 & horse        & 0.6132  \\
14 & motorbike    & 0.5846  \\
15 & person       & 0.1697  \\
16 & pottedplant  & 0.2828  \\
17 & sheep        & 0.6305  \\
18 & sofa         & 0.4661  \\
19 & train        & 0.5642  \\
20 & tvmonitor    & 0.4032  \\ \hline
\end{tabular}
\end{table}

\begin{table}[ht]
\centering
\caption{Overall Performance Metrics}
\begin{tabular}{|l|c|}
\hline
\textbf{Metric} & \textbf{Value} \\ \hline
Pixel Accuracy (pAcc) & 0.7981 \\
Mean Accuracy (mAcc)  & 0.6997 \\
Mean IoU (mIoU)       & 0.5003 \\ \hline
\end{tabular}
\end{table}

% The model achieved a mean IoU of 0.5003 on the PASCAL VOC dataset, with a pixel accuracy of 0.7981 and a mean accuracy of 0.6997. Performance varies significantly across classes: large and well-represented classes such as \textbf{background} (0.7784), \textbf{bus} (0.6753), \textbf{cat} (0.6591), and \textbf{dog} (0.6511) are segmented effectively, while smaller or less frequent classes like \textbf{person} (0.1697), \textbf{chair} (0.2525), and \textbf{potted plant} (0.2828) remain challenging. These results suggest that while the model captures the dominant classes well, further improvements are needed to better handle rare and small objects.

\begin{table}[!t]
    \centering
    \renewcommand{\arraystretch}{1.2}
    \setlength{\tabcolsep}{6pt}
    \begin{tabular}{l c c c c}
        \hline
        \textbf{Approach}                                                               & \textbf{Backbone}   & \textbf{Sup.} & \textbf{val}           & \textbf{test}          \\
        \hline
        \multicolumn{5}{c}{\textit{multi-stage weakly supervised approaches}}                                                    \\
        RCA$_{\text{CVPR'22}}$~\cite{wsss_RCA}             & ResNet101  & I+S  & 72.2          & 72.8          \\
        L2G$_{\text{CVPR'22}}$~\cite{wsss_L2G}             & ResNet101  & I+S  & 72.1          & 71.7          \\
        Mat-label$_{\text{ICCV'23}}$~\cite{wsss_MatLabel}  & ResNet101  & I+S  & 73.3          & \textbf{74.0} \\
        S-BCE$_{\text{ECCV'22}}$~\cite{wsss_s_bce}         & ResNet38   & I+S  & 68.1          & 70.4          \\
        RIB$_{\text{NeurIPS'21}}$~\cite{wsss_rib}          & ResNet38   & I    & 68.3          & 68.6          \\
        W-OoD$_{\text{CVPR'22}}$~\cite{wsss_ood}      & ResNet101  & I    & 69.8          & 69.9          \\
        ESOL$_{\text{NeurIPS'22}}$~\cite{wsss_esol}   & ResNet101  & I    & 69.9          & 69.3          \\
        VML$_{\text{IJCV'22}}$~\cite{wsss_vml}        & ResNet101  & I    & 70.6          & 70.7          \\
        AETF$_{\text{ECCV'22}}$~\cite{wsss_aetf}      & ResNet38   & I    & 70.9          & 71.7          \\
        MCTformer$_{\text{CVPR'22}}$~\cite{wsss_MCTformer} & ViT+Res38  & I    & 70.4          & 70.0          \\
        CDL$_{\text{IJCV'23}}$~\cite{wsss_cdl}        & ResNet101  & I    & 72.4          & 72.2          \\
        ACR$_{\text{CVPR'23}}$~\cite{wsss_acr}        & ViT        & I    & 72.4          & 72.4          \\
        BECO$_{\text{CVPR'23}}$~\cite{wsss_beco}      & MIT-B2     & I    & 73.7          & 73.5          \\
        FPR$_{\text{ICCV'23}}$~\cite{wsss_fpr}        & ResNet101  & I    & 70.0          & 70.6          \\
        USAGE$_{\text{ICCV'23}}$~\cite{wsss_usage}    & ResNet38   & I    & 71.9          & 72.8          \\
        CLIMS$_{\text{CVPR'22}}$~\cite{wsss_clims}    & ViT+Res101 & I+L  & 70.4          & 70.0          \\
        CLIP-ES$_{\text{CVPR'23}}$~\cite{wsss_clip_es}                       & ViT+Res101 & I+L  & \textbf{73.8} & 73.9          \\
        \hline
        \multicolumn{5}{c}{\textit{single-stage weakly supervised approaches}}                                                   \\
        1Stage$_{\text{CVPR'20}}$~\cite{wsss_single_stage}                   & ResNet38   & I    & 62.7          & 64.3          \\
        RRM$_{\text{AAAI'20}}$~\cite{wsss_reliability_does_matter}           & ResNet38   & I    & 62.6          & 62.9          \\
        AA\&AR$_{\text{ACMMM'21}}$~\cite{wsss_aaar}                                 & ResNet38   & I    & 63.9          & 64.8          \\
        SLRNet$_{\text{IJCV'22}}$~\cite{wsss_slr_net}                                  & ResNet38   & I    & 67.2          & 67.6          \\
        AFA$_{\text{CVPR'22}}$~\cite{wsss_afa_affinity_from_attention}                                     & MIT-B1     & I    & 66.0          & 66.3          \\
        TSCD$_{\text{AAAI'23}}$~\cite{wsss_tscd}                                    & MIT-B1     & I    & 67.0          & 67.5          \\
        ToCo$_{\text{CVPR'23}}$~\cite{wsss_toco_token_contrast}                                    & ViT        & I    & 71.1          & 72.2          \\
        WeCLIP (w/o CRF)                                                & ViT        & I+L  & 74.9          & 75.2          \\
        WeCLIP (w/ CRF)                                                 & ViT        & I+L  & \textbf{76.4} & \textbf{77.2} \\
        \hline
        \textbf{Ours - UniCL-AffSeg(w/oCRF)} & Swin-B & I+L & 50.0 & 50.0 \\
        \hline
    \end{tabular}
    \caption{
    Comparison of multi-stage and single-stage weakly supervised semantic segmentation methods on the PASCAL VOC 2012 validation and test sets (mIoU, \%). 
    Here, \textbf{I} denotes image-level supervision, \textbf{L} denotes language supervision, and \textbf{S} denotes saliency supervision. 
    Our method (\textbf{UniCL-AffSeg}) employs Swin-B as the backbone or image encoder. Without description, all have used Dense CRF during inference.
    }
    \label{tab:quantitative_results}
\end{table}

\subsection{Comparison with Exisiting Methods}
\label{subsec:comparison_with_baseline_methods}
Table~\ref{tab:quantitative_results} compares our proposed approach (UniCL) with both multi-stage and single-stage weakly supervised semantic segmentation (WSSS) methods. While state-of-the-art single-stage approaches such as WeCLIP achieve strong performance (76.4\%/77.2\% with CRF post-processing), our current UniCL-AffSeg implementation yields relatively modest results (50.0\% on both validation and test sets).  

It is important to note that UniCL-AffSeg follows a pipeline similar to WeCLIP, but with a critical distinction in how localization cues are obtained. Whereas WeCLIP relies on attention maps derived from a ViT backbone, UniCL-AffSeg leverages affinity maps computed from the Swin Transformer image encoder in combination with a ViT text encoder. Due to the window-based self-attention mechanism in Swin, the resulting feature representations tend to be more \textit{local} and less globally coherent than ViT attention maps. Consequently, the affinity maps fail to capture holistic object-level relationships, leading to weaker class activation maps (CAMs) at initialization. These initial CAMs often exhibit fragmented activations, particularly for large or complex objects, which limits the effectiveness of subsequent refinement stages.  

Furthermore, the feature maps extracted from the Swin image encoder, while rich in local details, lack the global semantic consistency necessary for reliable region affinity learning. This discrepancy explains why UniCL-AffSeg underperforms compared to WeCLIP, despite both methods operating within a broadly similar weakly supervised framework.  

% In summary, although UniCL-AffSeg adopts the same overarching design philosophy as WeCLIP, its reliance on Swin-based affinity maps rather than ViT-based attention maps hinders its ability to generate high-quality initial CAMs and to propagate semantic information effectively. These observations suggest that improving the global reasoning ability of the Swin image encoder, or integrating hybrid mechanisms that combine local affinity with global attention, may be essential for closing the performance gap with current state-of-the-art methods.





\newpage

\begin{figure}[H]
  \centering
  \setlength{\tabcolsep}{2pt} % adjust spacing
  \renewcommand{\arraystretch}{0.3}

  % Wrap the table in a colored box (requires \usepackage{tcolorbox})
  \begin{tcolorbox}[colframe=black!60, colback=white, boxrule=0.8pt, arc=2pt, left=2pt, right=2pt, top=2pt, bottom=2pt]
    \centering
    % CAMs with class labels on the left
    \begin{tabular}{m{2.5cm} c c c} % first column = label

    % Column headers
    & (a) Input & (b) WeCLIP & (c) Ours
    \\
    [1mm]

    {\textbf{Cat}}
    & \includegraphics[width=0.18\textwidth,height=0.18\textwidth]{figures/originals/2007_003778}
    & \includegraphics[width=0.18\textwidth,height=0.18\textwidth]{figures/val_cams/weclip/2007_003778_7}
    & \includegraphics[width=0.18\textwidth,height=0.18\textwidth]{figures/val_cams/ours/2007_003778_7}
    \\
    \textbf{Bicycle}
    & \includegraphics[width=0.18\textwidth,height=0.18\textwidth]{figures/originals/2011_000453}
    & \includegraphics[width=0.18\textwidth,height=0.18\textwidth]{figures/val_cams/weclip/2011_000453_1}
    & \includegraphics[width=0.18\textwidth,height=0.18\textwidth]{figures/val_cams/ours/2011_000453_1}
    \\
    \textbf{Bird}
    & \includegraphics[width=0.18\textwidth,height=0.18\textwidth]{figures/originals/2011_001902}
    & \includegraphics[width=0.18\textwidth,height=0.18\textwidth]{figures/val_cams/weclip/2011_001902_2}
    & \includegraphics[width=0.18\textwidth,height=0.18\textwidth]{figures/val_cams/ours/2011_001902_2}
    \\
    \textbf{Boat}
    & \includegraphics[width=0.18\textwidth,height=0.18\textwidth]{figures/originals/2010_003599}
    & \includegraphics[width=0.18\textwidth,height=0.18\textwidth]{figures/val_cams/weclip/2010_003599_3}
    & \includegraphics[width=0.18\textwidth,height=0.18\textwidth]{figures/val_cams/ours/2010_003599_3}
    \\
    \textbf{Pottedplant}
    & \includegraphics[width=0.18\textwidth,height=0.18\textwidth]{figures/originals/2011_000145}
    & \includegraphics[width=0.18\textwidth,height=0.18\textwidth]{figures/val_cams/weclip/2011_000145_15}
    & \includegraphics[width=0.18\textwidth,height=0.18\textwidth]{figures/val_cams/ours/2011_000145_15}
    \\
    \textbf{Car}
    & \includegraphics[width=0.18\textwidth,height=0.18\textwidth]{figures/originals/2010_005119}
    & \includegraphics[width=0.18\textwidth,height=0.18\textwidth]{figures/val_cams/weclip/2010_005119_6}
    & \includegraphics[width=0.18\textwidth,height=0.18\textwidth]{figures/val_cams/ours/2010_005119_6}
    \\
    \textbf{Bus}
    & \includegraphics[width=0.18\textwidth,height=0.18\textwidth]{figures/originals/2010_000148}
    & \includegraphics[width=0.18\textwidth,height=0.18\textwidth]{figures/val_cams/weclip/2010_000148_5}
    & \includegraphics[width=0.18\textwidth,height=0.18\textwidth]{figures/val_cams/ours/2010_000148_5}
    \\
    \textbf{Person}
    & \includegraphics[width=0.18\textwidth,height=0.18\textwidth]{figures/originals/2007_005702}
    & \includegraphics[width=0.18\textwidth,height=0.18\textwidth]{figures/val_cams/weclip/2007_005702_14}
    & \includegraphics[width=0.18\textwidth,height=0.18\textwidth]{figures/val_cams/ours/2007_005702_14}
  \end{tabular}

  \end{tcolorbox}

  \caption{Qualitative comparison of CAMs between WeCLIP and our UniCL-AffSeg on PASCAL VOC 2012 \textit{val} set.}
  \label{fig:qualitative_comparison_cam_val}
\end{figure}



\begin{figure}[H]
  \centering
  \setlength{\tabcolsep}{2pt} % adjust spacing
  \renewcommand{\arraystretch}{0.3}

  % Wrap the table in a colored box (requires \usepackage{tcolorbox})
  \begin{tcolorbox}[colframe=black!60, colback=white, boxrule=0.8pt, arc=2pt, left=2pt, right=2pt, top=2pt, bottom=2pt]
    \centering
    % CAMs with class labels on the left
    \begin{tabular}{m{2.5cm} c c c} % first column = label

    % Column headers
    & (a) Input & (b) WeCLIP & (c) Ours
    \\
    [1mm]

    {\textbf{Motorbike}}
    & \includegraphics[width=0.18\textwidth,height=0.18\textwidth]{figures/originals/2007_002260}
    & \includegraphics[width=0.18\textwidth,height=0.18\textwidth]{figures/test_cams/weclip/2007_002260_13}
    & \includegraphics[width=0.18\textwidth,height=0.18\textwidth]{figures/test_cams/ours/2007_002260_13}
    \\
    \textbf{Aeroplane}
    & \includegraphics[width=0.18\textwidth,height=0.18\textwidth]{figures/originals/2007_000033}
    & \includegraphics[width=0.18\textwidth,height=0.18\textwidth]{figures/test_cams/weclip/2007_000033_0}
    & \includegraphics[width=0.18\textwidth,height=0.18\textwidth]{figures/test_cams/ours/2007_000033_0}
    \\
    \textbf{Train}
    & \includegraphics[width=0.18\textwidth,height=0.18\textwidth]{figures/originals/2007_000123}
    & \includegraphics[width=0.18\textwidth,height=0.18\textwidth]{figures/test_cams/weclip/2007_000123_18}
    & \includegraphics[width=0.18\textwidth,height=0.18\textwidth]{figures/test_cams/ours/2007_000123_18}
    \\
    \textbf{Dog}
    & \includegraphics[width=0.18\textwidth,height=0.18\textwidth]{figures/originals/2007_003194}
    & \includegraphics[width=0.18\textwidth,height=0.18\textwidth]{figures/test_cams/weclip/2007_003194_11}
    & \includegraphics[width=0.18\textwidth,height=0.18\textwidth]{figures/test_cams/ours/2007_003194_11}
    \\
    \textbf{Car}
    & \includegraphics[width=0.18\textwidth,height=0.18\textwidth]{figures/originals/2007_006277}
    & \includegraphics[width=0.18\textwidth,height=0.18\textwidth]{figures/test_cams/weclip/2007_006277_6}
    & \includegraphics[width=0.18\textwidth,height=0.18\textwidth]{figures/test_cams/ours/2007_006277_6}
    \\
    \textbf{Bird}
    & \includegraphics[width=0.18\textwidth,height=0.18\textwidth]{figures/originals/2007_001289}
    & \includegraphics[width=0.18\textwidth,height=0.18\textwidth]{figures/test_cams/weclip/2007_001289_2}
    & \includegraphics[width=0.18\textwidth,height=0.18\textwidth]{figures/test_cams/ours/2007_001289_2}
    \\
    \textbf{Person}
    & \includegraphics[width=0.18\textwidth,height=0.18\textwidth]{figures/originals/2007_000783}
    & \includegraphics[width=0.18\textwidth,height=0.18\textwidth]{figures/test_cams/weclip/2007_000783_14}
    & \includegraphics[width=0.18\textwidth,height=0.18\textwidth]{figures/test_cams/ours/2007_000783_14}
    \\
    \textbf{Chair}
    & \includegraphics[width=0.18\textwidth,height=0.18\textwidth]{figures/originals/2007_005844}
    & \includegraphics[width=0.18\textwidth,height=0.18\textwidth]{figures/test_cams/weclip/2007_005844_8}
    & \includegraphics[width=0.18\textwidth,height=0.18\textwidth]{figures/test_cams/ours/2007_005844_8}
    \\
  \end{tabular}

  \end{tcolorbox}

  \caption{Qualitative comparison of CAMs between WeCLIP and our UniCL-AffSeg on PASCAL VOC 2012 \textit{test} set.}
  \label{fig:qualitative_comparison_cam_test}
\end{figure}


%summary
\subsection{Summary of Quantitative Analysis}
\label{subsubsec:quantitative_summary}

Overall, the quantitative results reveal that our UniCL-AffSeg framework demonstrates reasonable segmentation ability under image- and language-level supervision, despite operating without pixel-level annotations. The model achieves comparable performance on both validation (50.3\% mIoU) and test (50.8\% mIoU) sets, indicating good generalization and stable behavior across unseen data. 

A closer look at per-class IoUs shows that UniCL-AffSeg performs strongly on large, visually distinct categories such as \textit{bus}, \textit{cow}, and \textit{sheep}, while struggling on small or underrepresented ones like \textit{person}, \textit{chair}, and \textit{bicycle}. This disparity highlights the limitations of Swin-based affinity representations in capturing fine-grained or globally contextual features. When compared with state-of-the-art weakly supervised segmentation models, our approach trails methods like WeCLIP and ToCo, which rely on ViT-based attention mechanisms. The relatively lower mIoU suggests that while Swin Transformers provide strong local representations, their limited global context impedes effective region propagation during pseudo-label generation. 

These findings collectively emphasize the need for integrating global reasoning components—such as cross-window attention or hybrid ViT-Swin architectures—to bridge the gap with ViT-based WSSS models and improve small-object segmentation consistency.

